%_______________________________________________________________________________
% main.tex

\input{preamble12.tex}
\hypersetup{%
    pdfauthor={Mike Pierce}%
   ,pdftitle={Pop Quiz | Math 113}%
   ,pdfkeywords={Pierce,CMU,Colorado,College Algebra,113}%
   ,pageanchor=false%
}
\geometry{
    margin=1in%
   ,left=0.75in%
   ,right=0.75in%
}
\usepackage{fourier}
\usepackage[default]{comicneue}
\input{accessible-colors.tex}
\input{newcommand.tex}
\input{newenvironment.tex}
\pagenumbering{gobble}
\usepackage{pagegrid}
\pagegridsetup{
    step=0.5in
    ,firstcolor=blue
    ,secondcolor=orange
    ,foreground=true
    ,disable
}


\begin{document}

\begin{center}
    {\Huge{Pop Quiz}}
    \\ Math 113-001/6 College Algebra
    \\ Colorado Mesa University Fall 2022
\end{center}

\vspace{0.5in-2px}
%\vspace{2em}
%\textsc{Name}: \enspace \hrulefill
%\vspace{1em}

\begin{enumerate}

    \item 
        
        The following expression is equal to \(z^\square\)
        for some number \(\square\).
        What number must \(\square\) be?

        \(\displaystyle \qquad\frac{\sqrt{\left(z^3\right)^3z^8}}{z^5} \)
        \vfill\null

    \item 
        What number is \(\log_7\left(\ex^3\right)\) equal to?
        Write this number as a decimal accurate to five digits,
        \vfill\null

    \item 
        If \(\log_3(x) = 7\), 
        what number must \(\log_3\left(81x^2\right)\) be equal to?
        \vfill\null
        \vfill\null

        \newpage

    \item 
        The percent of Americas who are medically classified as \emph{obese}
        from 1990 projected through 2030 can be modeled
        by \(f(t) = -31.75 + 18.5 \ln(t)\), 
        with \(t\) equal to the number of years after the year 1980.
        \begin{enumerate}
            \item
                What percentage of Americans would currently be classified as obese
                according to this model?
                \vfill
            \item
                Does this model suggest obesity is become more frequent
                or less frequent among Americans?
                \vfill
            \item
                The fact that \(t\) is the number of years after 1980
                when the domain of the model starts at 1990 is a tad silly.
                How would the formula \(f(t)\) have to change
                if instead \(t\) were the number of years after 1990?
                \vfill
            \item
                This model is proposed to become inaccurate beyond the year 2030.
                Ignoring that limitation,
                for what year does this model predict
                that \emph{every single American} will be obese?
                \vfill
                \vfill
        \end{enumerate}

\end{enumerate}

\end{document}

