%_______________________________________________________________________________
% main.tex

\input{preamble12.tex}
\hypersetup{%
    pdfauthor={Mike Pierce}%
   ,pdftitle={Pop Quiz | Math 113}%
   ,pdfkeywords={Pierce,CMU,Colorado,College Algebra,113}%
   ,pageanchor=false%
}
\geometry{
    margin=1in%
   ,left=0.75in%
   ,right=0.75in%
}
\usepackage{fourier}
\usepackage[default]{comicneue}
\input{accessible-colors.tex}
\input{newcommand.tex}
\input{newenvironment.tex}
\pagenumbering{gobble}
\usepackage{pagegrid}
\pagegridsetup{
    step=0.5in
    ,firstcolor=blue
    ,secondcolor=orange
    ,foreground=true
    ,disable
}


\begin{document}

\begin{center}
    {\Huge{Pop Quiz}}
    \\ Math 113-001/6 College Algebra
    \\ Colorado Mesa University Fall 2022
\end{center}

\vspace{0.5in-2px}
%\vspace{2em}
%\textsc{Name}: \enspace \hrulefill
%\vspace{1em}

\begin{enumerate}

    \item 
        Below is a table reporting 
        the US national debt\footnote{\href{https://fiscaldata.treasury.gov/datasets/historical-debt-outstanding/historical-debt-outstanding}{\texttt{fiscaldata.treasury.gov/datasets/historical-debt-outstanding}}},
        presented in \emph{trillions} of dollars, for select years.
        \par\begin{center}\begin{tabular}{r||c|c|c|c|c|c}
            year                & 1992 & 2001 & 2007 & 2011 & 2017 & 2021 \\\hline
            debt (in trillions) & 4.06 & 5.81 & 9.00 & 14.79 & 20.24 & 28.43 
        \end{tabular}\end{center}
        \begin{enumerate}
            \item
                Do you think an exponential model 
                or a logarithmic model would fit the data best?
                \vfill\null
            \item
                Based on your choice in the previous part, 
                perform regression to find a function 
                of \(t\) years after 1990 that models the data.
                Write this function below with parameters (\(a\) and \(b\))
                rounded to two decimal places.
                (If you do not have a calculator capable of regression,
                simply write ``no calc''
                and circle whichever of these functions you think 
                fits the data best.)
                Use this function as your model 
                for the remaining questions.
                \par\(\displaystyle 3(1.07)^t \quad -4+7\ln\left(t\right) \quad 4(0.92)^t \quad 2+4.06\ln\left(t\right)\)
            \item
                What does your model predict 
                the national debt this year to be?
                \vfill\null
                \vfill\null
            \item
                The actual national debt this year is
                about 30.93 trillion dollars.
                What is the difference between this figure 
                and your model's prediction?
                \vfill\null
                \vfill\null
            \item
                According to your model, 
                when will the national debt be
                one quadrillion (1000 trillion) dollars?
                % \log_{1.0721}\frac{1000}{3.1146}  2073
                \vfill\null
                \vfill\null
                \vfill\null
                \vfill\null
        \end{enumerate}

        \newpage

    %\item 
    %    Annuities
    %    \vfill\null

    \item 
        Coloramo Credit Union offers \emph{Certificates of Deposit} (CDs)
        to its account holders\footnote{\href{https://www.coloramo.org/rates/\#certificate}{\texttt{coloramo.org/rates/\#certificate}}}.
        A CD is a special account where you deposit 
        a certain amount of cash and promise
        not to withdraw that cash for a pre-determined amount of time.
        In return for this promise the credit union offers you a higher interest rate
        than they do for a typical savings account.
        \begin{enumerate}
            \item
                Coloramo offers a four-year CD with a 1.837\% interest rate
                compounded quarterly (once every three months).
                If you invest \$1000 into this CD,
                how much will your CD be worth after those four years?
                \vfill\null
            \item
                How much more would the CD be worth 
                if instead of compounding the interest quarterly
                Coloramo compounded the interest every month?
                \vfill\null
            \item
                How much more would the CD be worth 
                if instead of compounding the interest quarterly
                Coloramo compounded the interest \emph{continually}?
                \vfill\null
            \item
                Suppose you plan on graduating college in four years.
                You're living on campus now, so you don't have a car, 
                but you know you'll need a car to commute to work after you graduate.
                You estimate that \$21,000 should be enough to buy an adequate car
                in four years when you graduate, and figure why not
                deposit some money into Coloramo's four-year CD 
                and collect that \$21,000 in four years.
                How much money should you deposit in Coloramo's four year CD
                such that after four years it will be worth \$21,000?
                \vfill\null
                \vfill\null
                \vfill\null
        \end{enumerate}

\end{enumerate}

\newpage

%\begin{center}
%    {\Huge{Pop Quiz Five}}
%    \\ Math 113-001/6 College Algebra
%    \\ Colorado Mesa University Fall 2022
%\end{center}
%
%\vspace{0.5in-2px}

The table gives the life expectancy (life span)
for people in the United States for select birth years
from 1920 and through 2018\footnote{\href{https://www.cdc.gov/nchs/data-visualization/mortality-trends/}{cdc.gov/nchs/data-visualization/mortality-trends}}.
%from 1920 and through 2020\footnote{Pearson: National Center for Health Statistics}.

\begin{minipage}[t]{0.8\textwidth}
    \begin{enumerate}
        \item
    Using technology, perform regression to
            find a logarithmic function \(f(t) = a + b\ln(t)\) 
    that models the data,
    with \(t\) equal to \(0\) in 1900.

        \item
    Using technology, perform regression to
    find a power model \(g(t) = a\cdot b^t\) 
    that models the data,
    with \(t\) equal to \(0\) in 1900.

        \item
            Using technology,
    plot the graphs of each of these functions
    along with the data on the same set of axis
    and decide which model fits the data best. 
    On your plot make sure the domain of \(t\)
    matches up with the years 1920--2030.

\item 
    What does the model that you decided 
            fits the data best predict your own life expectancy to be?

\item 
    What does the model that you decided 
            fits the data best predict 
            the life expectancy of a baby born today to be?

\item 
    What year does the logarithmic model
    predict will be the year life expectancy is 80 years?

\item 
    In the years since 2018, 
    life expectancy in the US has started to decline\footnotemark.
            This is suspected to be due to the \textsc{COVID}-19 pandemic.
            Someone born in the US in 2019, 2020, and 2021 
            has a life expectancy of 79 years, 77 years, and 76.1 years respectively.
            Add this new data to the data set.

            Both logarithmic and power functions are increasing functions.
            Since life expectancy is decreasing in recent years though,
            these functions may no longer provide the most accurate models.
            What's a type of function that more accurately models
            data that initially increases, but then begins to decrease?
            Using technology, perform regression to
            find such function \(h(t)\) 
            that models the data,
            again with \(t\) equal to \(0\) in 1900.

\item 
    What does this new model predict
            the life expectancy of a baby born today to be?
            How does this compare to the figure predicted
            by either your logarithmic or power model?

    \end{enumerate}

\end{minipage}
\footnotetext{\href{https://www.npr.org/sections/health-shots/2022/08/31/1120192583/}{npr.org/sections/health-shots/2022/08/31/1120192583}}.
\hspace{1em}
\begin{minipage}[t]{0.19\textwidth}
\par\begin{center}\begin{tabular}{l|l}
Year & Life Span \\\hline
    1920 & 54.1 \\
    1930 & 59.7 \\
    1940 & 62.9 \\
    1950 & 68.2 \\
    1960 & 69.7 \\
    1970 & 70.8 \\
    1975 & 72.6 \\
    1980 & 73.7 \\
    1987 & 75.0 \\
    1988 & 74.9 \\
    1989 & 75.2 \\
    1990 & 75.4 \\
    1992 & 75.8 \\
    1994 & 75.7 \\
    1996 & 76.1 \\
    1998 & 76.7 \\
    1999 & 76.7 \\
    2000 & 76.8 \\
    2001 & 77.2 \\
    2002 & 77.0 \\
    2003 & 77.6 \\
    2004 & 77.5 \\
    2005 & 77.6 \\
    2010 & 78.7 \\
    2015 & 78.7 \\
    2016 & 78.7 \\
    2017 & 78.6 \\
    2018 & 78.7
\end{tabular}\end{center}
\end{minipage}

\end{document}

